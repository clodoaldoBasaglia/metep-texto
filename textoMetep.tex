\documentclass{article}
\usepackage[utf8]{inputenc}

\title{Construção de um VANT autônomo remotamente controlado utilizando-se de GPS e mapas georreferenciados}
\author{Clodoaldo A. Basaglia da Fonseca }
\date{Maio de 2017}

\usepackage{natbib}
\usepackage{graphicx}

\begin{document}

\maketitle

\section{Introdução}
Veiculos Aereos Não tripulados vem sendo utilizados a décadas em vários ambientes distintos, desde operações militares durante conflitos armados, no sensioramento de grandes quantidades de terra, na segurança de grandes eventos, bem como na manutenção da soberania das fronteiras das nações. Veículos dessa classe datam da Segunda Guerra(1939-1945), onde eram utilizados para treinamento. Com o passar dos tempos e avanço das tecnologias utilizadas em sua construção, o papel dos VANT's vem se modificando, ampliando o espectro de atividades onde podem ser utilizados. Para tal, novos metodos de controle que visão a qualidade de voo e a facilidade de operação, bem como a confiabilidade são necessários. Propõe-se também, que, dado o aumento do poder de processamento dos circuitos embarcados atuais, seja possivel que um desdes aparelhos seja capaz de se manter em voo por longas horas sem a necessidade de intervenção humana, no que se diz respeito ao seu controle, salvo em determinadas atividades. Com o advento de tecnicas de geoprocessamento e sensoriamento, existem uma vasta gama de mapas georreferenciados que possibilitam a navegabilidade de VANT's pelo espaço aéreo com certa confiabilidade, dado o desenvolvimento de algoritmos que possam fornecer coordenadas em tempo real para o aparelho.



\end{document}
